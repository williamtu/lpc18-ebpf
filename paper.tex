\documentclass[10pt]{sigplanconf}

\usepackage{graphicx}
\usepackage{url}
\usepackage{listings}
\usepackage{amsmath}
\usepackage[font={small}]{caption}
\newcommand{\mycomment}[1]{}
\newcommand{\dfn}[1]{\textit{#1}} % For initial definitions of terms
\newcommand{\myfig}[3]
 {
 \begin{figure}
 \centerline{\includegraphics[width=#2]{graph/#1.pdf}}
 \caption{\sl \small #3}
 \label{#1:fig}
 \vspace*{-0.15in}
 \end{figure}
 }
\newcommand{\myfigcross}[3]
{
\begin{figure*}[ht]
\centerline{\includegraphics[width=#2]{graph/#1.pdf}}
\caption{\sl \small #3}
\label{#1:fig}
\vspace*{-0.15in}
\end{figure*}
}

% For marking text we're probably dropping.
\newcommand{\ignore}[1]{}

% For marking text that we might like to include in a final draft, if
% accepted to RADIO, but that will not fit within the 5-page limit.
\newcommand{\saveforlater}[1]{}

% For marking important work items.
\newcommand{\todo}[1]{\textbf{TODO: #1}}

\lstset{ %
    captionpos=b,
    frame=single
}

\begin{document}
\title{Bringing the Power of eBPF to Open vSwitch}
\authorinfo{
  \(
  \begin{matrix}
    % Alphabetical order.
    \textrm{William Tu} & \textrm{Joe Stringer} & \textrm{Yifeng Sun} & \textrm{Yi-Hung Wei}\\
    \textrm{u9012063@gmail.com} & \textrm{stringerjoe@vmware.com} & \textrm{pkusunyifeng@gmail.com} & \textrm{yihung.wei@gmail.com} \\
  \end{matrix}
  \)
}
{VMware Inc.}{}

\maketitle

\begin{abstract}
In this paper we discuss two projects, the OVS-eBPF project, with
the goal of re-writing existing openvswitch kernel module's feature
into eBPF code at TC attachment hook.
And the OVS-AFXDP project, where we 


\end{abstract}

\section{Introduction}\label{introduction}
Among the various ways of using eBPF, OVS has been exploring the power of eBPF in three:
(1) attaching eBPF to TC, (2) offloading a subset of processing to XDP, and (3) by-passing
the kernel using AF\_XDP. Unfortunately, as of today, none of the three approaches satisfies
the requirements of OVS. In this paper, we share the challenges we faced, experience learned,
and seek for feedbacks from the community for future direction.

Attaching eBPF to TC started first with the most aggressive goal: we planned to re-implement
the entire features of OVS kernel datapath under net/openvswitch/* into eBPF code.
We worked around a couple of limitations, for example, the lack of TLV support led us to
redefine a binary kernel-user API using a fixed-length array; and without a dedicated way to
execute a packet, we created a dedicated device for user to kernel packet transmission,
with a different BPF program attached to handle packet execute logic.
Currently, we are working on connection tracking.
Although a simple eBPF map can achieve basic operations of conntrack table lookup and commit,
how to handle NAT, (de)fragmentation, and ALG are still under discussion.

Moving one layer below TC is called XDP (eXpress Data Path), a much faster layer for packet
processing, but with almost no extra packet metadata and limited BPF helpers support.
Depending on the complexity of flows, OVS can offload a subset of its flow processing to XDP
when feasible. However, the fact that XDP has fewer helper function support implies that either
1) only very limited number of flows are eligible for offload, or 2) more flow processing logic
needed to be done in native eBPF.
 
AF\_XDP is another form of XDP but providing a socket interface for control plane and a shared
memory API for accessing packets from userspace application. OVS today has another full-fledged
datapath implementation in userspace, called dpif-netdev, used by DPDK community.
By treating the AF\_XDP as a fast packet-I/O channel, the OVS dpif-netdev can satisfy most of the
features. We are working on building the prototype and evaluating its performance.


Existing OVS consists of three datapath implementations:
dpif-netlink
dpif-netdev
dpif-windows
In this paper, OVS-eBPF project adds a new datapath type
dpif-bpf
and OVS-AFXDP project reuses the dpif-netdev interface, with
linux netdev receiving/sending packets from AFXDP sockets.  

\section{Background}

\subsection{OVS Forwarding Model}

\begin{figure}
{\scriptsize
\begin{verbatim}
              
                +-------------------+
 slow path      |    ovs-vswitchd   |
                +-----^-------------+
                      |       |flow installation
----------------------|-------|-----------------
 fast path     upcall |       | 
 (datapath)  +----------------v---------+ 
  packets--> | parse, lookup, actions   |  --> output
             +--------------------------+
\end{verbatim}
}
\vspace{-1.0em}
\caption{An example of mirrored packet with outer header containing
the GRE and ERSPAN header, followed by the innter Ethernet frame.}
\label{erspanhdr}
\vspace{-1.0em}
\end{figure}

OVS is widely used in virtualized data center environments as a software
switching layer inside various operating systems, including FreeBSD,
Windows Hyper-V, Solaris and Linux. As shown in
Figure~\ref{ovsintro:fig}, the architecture of OVS consists of two major
components: a slow path and a fast path. OVS begins processing packets
in its datapath, the fast path, shortly after the packet is received by the NIC in the host OS.
The OVS datapath first performs packet parsing to extract relevant protocol
headers from the packet and store it locally in a manner that is efficient for
performing lookups (flow key), then it uses this information to look into the
match/action cache (flow table) and determines what needs to be done for this
packet. If there is no match in the flow table, the datapath passes the packet
from the kernel up to the slow path, \verb+ovs-vswitchd+, which maintains the
full determination of what needs to be executed to modify and forward
the packet correctly.  This process is called packet {\em upcall} and
usually happens at the first packet of a flow seen by the OVS datapath.
If the packet matches in this flow table, then the OVS datapath executes its
corresponding actions from the flow table lookup result and updates its flow
statistics.

In this model, the \verb+ovs-vswitchd+ determines how the packet should be handled,
and passes the information to the datapath inside the kernel using a
Linux generic netlink interface.  Over the years the OVS datapath features evolved.
The initial OVS datapath used a microflow cache for its flow table,
essentially caching exact-match entries for each
transport layer connection's forwarding decision.  And in later versions,
two layers of caching were used: a microflow cache and a megaflow cache,
which caches forwarding decisions for traffic aggregates beyond individual
connections.  In recent versions of OVS, datapath implementations include
features such as connection tracking, stateful network address translation, and
support for layer 3 tunneling protocols.

\mycomment{
As each new feature in OVS datapath requires deep knowledge and programming
paradigm of the target operating system, e.g., how Linux packets are received,
how Windows kernel module does firewalling, and each operating system
community has its own software upstream and maintenance processes, the software
development cycle of a new datapath feature becomes complex and requires
detailed platform-specific knowledge. Each platform introduces its own
development and maintenance burden, and it is common for new features to take
six months or more before implementations are available across all datapaths.
As a result, providing a programmable OVS datapath which allows extending the
features flexibly with lower maintenance cost becomes a critical requirement.
}

\subsection{eBPF Basics}\label{sec:ebpf}
Berkeley Packet Filter, BPF, is an instruction set architecture
proposed by Steven McCanne and Van Jacobson in 1993~\cite{cbpf}.  
BPF was designed as a generic packet filtering solution and is widely
used by every network operator today, through the well-known tcpdump/wireshark
applications. A BPF interpreter is attached early in the packet receive call
chain, and it executes a BPF program as a list of instructions.
A BPF program typically parses a packet and decides
whether to pass the packet to a userspace socket.  With its simple architecture
and early filtering decision logic, it can execute this logic efficiently.

%P2 from cBPF to eBPF
For the past few years, the Linux kernel community has improved the traditional
BPF (now renamed to classic BPF, cBPF) interpreter inside the kernel with
additional instructions, known as extended BPF (eBPF). eBPF was introduced with
the purpose of broadening the programmability of the Linux kernel. Within the
kernel, eBPF instructions run in a virtual machine environment. The virtual
machine provides a few registers, stack space, program counter, and a way to
interact with the rest of the kernel through a mechanism called helper
functions.  Similar to cBPF, eBPF operates in an event-driven model on a
particular hook point; each hook point has its own execution {\em context} and
execution at the hook point only starts when a particular type of event fires.
A BPF program is written against a specific context. For example, a BPF program
attached to a raw socket interface has a context which includes the packet, and
the program is only triggered to run when there is an incoming packet to the
raw socket.

% eBPF map, helper, tail call and syscall
%helper
The eBPF virtual machine provides a completely isolated environment for its
bytecode running inside; in other words, it cannot arbitrarily call other
kernel functions or access into memory outside its own environment. To interact
with the outside world, the eBPF architecture white-lists a set of helper
functions that a BPF program can call, depending on the {\em context} of the
BPF program.  For example, a BPF program attached to raw socket in a packet
context could invoke VLAN push or pop related helper functions, while a BPF
program with a kernel tracing context could not. %The complete list of helper
%functions can be found at \verb+linux/include/uapi/linux/bpf.h+.

%map
To store and share state, eBPF provides a mechanism to interact with a variety
of key/value stores, called \textit{maps}. eBPF maps reside in the kernel, and can be
shared and accessed from eBPF programs and userspace applications. eBPF
programs can access maps through helper functions, while userspace applications
can access maps through BPF system calls. There are a variety of map types for
different use cases, such as hash tables or arrays. These are created by a
userspace program and may be shared between multiple eBPF programs running in
any hook point.

%tail call
Finally, eBPF tail call~\cite{tailcall} is a mechanism allowing one eBPF program to trigger execution of 
another eBPF program, providing users the flexibility of composing a chain of
eBPF programs with each one focusing on particular features.  Unlike a
traditional function call, this tail call mechanism calls another program
without returning back to the caller's program. The tail call reuses the
caller's stack frame, which allows the eBPF implementation to minimize call
overhead and simplifies verification of eBPF programs.
% end of bpf intro

\subsection{XDP: eXpress Data Path}
\myfig{tcebpf}{2.6in}{The workflow of TC and XDP eBPF development process and its
packet flow.  The eBPF program compiled by LLVM+clang is loaded into the kernel
using iproute. The kernel runs the program through a verification stage, and
subsequently attaches the program to the TC/XDP ingress hook point.  Once
successfully loaded, an incoming packet received by XDP/TC ingress will execute the
eBPF program.}

There are several hook points where eBPF programs may be attached in recent
Linux kernels. XDP is still an eBPF program, but its attachment point is at the
lowest level of the network stack.  Due to its unique attachment point, the XDP
has its own {\em context}; the input parameters to the XDP eBPF program and return
values have different meanings than the TC eBPF program. XDP shows performance
closed to line rate of the device, because the XDP program is triggered immediately
at the network device driver's packet receiving code path.
Due to its lowest hook point at the networking stack, XDP input parameter has only
pointer to the beginning and end of the packet data plus a few metadatas.
XDP also supports accessing to eBPF maps and tail calls, but with much
less number of helper functions than the eBPF program at TC hook.

Figure~\ref{tcebpf:fig}
shows the typical workflow for installing an eBPF program to the TC/XDP hook point,
and how packets trigger eBPF execution.  Clang and LLVM takes a program
written in C and compiles it down to the eBPF instruction set, then emits an
ELF file that contains eBPF instructions.  An eBPF loader, such as iproute,
takes the ELF file, parses the programs and maps information from it and
issues BPF syscalls to load the program.  If it passes the BPF verifier,
then the program is attached to the hook point (in this case, TC/XDP), and
subsequent packets through the TC/XDP ingress hook will trigger execution of the
eBPF programs.

\subsection{AF\_XDP Socket}
% afxdp
AF\_XDP is a new Linux address family that aims for high packet I/O
performance. Traditionally, a userspace program receives packets from
kernel through the socket API.  By creating a socket with address family
such as AF\_PACKET, the userspace program can receive/send the raw packets
at the device driver layer.  Although the AF\_PACKET family has been using
in many places such as tcpdump, its performance does not catch up the the
recent high speed network devices, such as 40G/100G NICs.
Performance evaluation~\cite{danialafpacket,johnafxdp} of AF\_PACKET
shows less than 2 million packets per second using single core.

AF\_XDP was proposed and upstreamed to Linux kernel since 4.18~\cite{afxdp} 
The core idea behind the AF\_XDP is to leverage the XDP eBPF program's
early access to the raw packet, and create a high speed channel from the XDP to
a userspace socket interface. In other word, AF\_XDP socket family connects the
XDP packet receving/sending path to the user space, bypassing the rest of the
Linux networking stacks.
An AF\_XDP socket, called XSK, is create using the normal socket() system
call. Unlike AF\_PACKET using the send() and receive() syscalls,
XSK introduces two rings in userspace: the RX ring and the TX ring.
The userspace program using XSK needs to properly configure and maintain
the RX and TX ring structure in order to receive and send packets.

For example, to receive packets, a set of RX descriptors pointing to empty
packet buffer needs to be filled into the RX ring.  When packets arrives,
the userspace program fetches the packet data from the desciptors, and
refill the empty buffer to the RX ring structure, in order to receive new
incoming packets.  For sending packets, a set of descriptors pointing the
packet buffer with contents is filled to the TX ring (FILL queue).
Then, the userspace
program issues send() system call.  It is up to the userspace program to
make sure whether the packets have been sent or not, by checking the TX
ring structure.

% concept
Application still HW agnostic with ZC
Each application gets its own packet buffer and TX/RX descriptor rings
Packet buffers can be shared if desired
  TX/RX descriptor rings always private to process
% AF\_XDP has 3 mode
% zero copy, native mode,
XDP\_SKB:
– Works on any netdevice using sockets and generic XDP path
XDP\_DRV:
– Works on any device with XDP support (all three NDOs)
XDP\_DRV + ZC:
– Need buffer allocator support in driver + a new NDO for TX
All the packet data buffers used in TX/RX ring are allocated from a specific
memory area called UMEM which consists of a number of fixed size chunks.
A descriptor in RX/TX ring points to the element in UMEM by its address.
The address is simply an offset within the entire UMEM memory region.
The UMEM also has two rings: the FILL ring and the COMPLETION ring.
The Fill ring is used by the application to send down addr for the kernel
to fill in with RX packet data. References to these frames will then
appear in the RX ring once each packet has been received. The 
completion ring, on the other hand, contains frame addr that the 
kernel has transmitted completely and can now be used again by user
space, for either TX or RX. 

Thus, the frame addrs appearing in the 
completion ring are addrs that were previously transmitted using the 
TX ring. In summary, the RX and FILL rings are used for the RX path
and the TX and COMPLETION rings are used for the TX path.


\section{OVS eBPF Datapath}
\mycomment{
\begin{figure}
{\scriptsize
\begin{verbatim}

               _
              |   +-------------------+
              |   |    ovs-vswitchd   |
              |   +-------------------+
    userspace |   |      ofproto      |<-->OpenFlow controllers
              |   +--------+-+--------+  
              |   | netdev | |ofproto-|
              |   +--------+ |  dpif  |
                  | netdev | +--------+
   *eBPF hook --> |provider| |  dpif  |
                  +---||---+ +--------+
              |       ||     |  dpif  | <--- *eBPF provider
              |       ||     |provider|
              |_      ||     +---||---+
                      ||         ||
               _  +---||-----+---||---+
              |   |          |datapath| <--- *eBPF datapath
       kernel |   |          +--------+
              |   |                   |
              |_  +--------||---------+
                           ||
                        physical
                           NIC
\end{verbatim}
}
\vspace{-1.0em}
\caption{An example of mirrored packet with outer header containing
the GRE and ERSPAN header, followed by the innter Ethernet frame.}
\label{erspanhdr}
\vspace{-1.0em}
\end{figure}
\subsection{Tunnel Support}
\subsection{Megaflow Support}
\subsection{Connection Tracking Support}
}

\section{Userspace OVS with AF\_XDP}
\begin{figure}
{\scriptsize
\begin{verbatim}
              |   +-------------------+
              |   |    ovs-vswitchd   |<-->ovsdb-server
              |   +-------------------+
              |   |      ofproto      |<-->OpenFlow controllers
              |   +--------+-+--------+ 
              |   | netdev | |ofproto-|
    userspace |   +--------+ |  dpif  |
              |   | AF_XDP | +--------+
              |   | netdev | |  dpif  |
              |   +---||---+ +--------+
              |       ||     |  dpif- |
              |       ||     | netdev |
              |_      ||     +--------+  
                      ||         
               _  +---||-----+--------+
              |   | XDP program +     |
 Linux kernel |   |   eBPF map        |
              |_  +--------||---------+
                           ||
                        physical
                           NIC
\end{verbatim}
}
\vspace{-1.0em}
\caption{OVS Architecture AF\_XDP}
\label{ovsafxdp}
\vspace{-1.0em}
\end{figure}

\subsection{Datapath and Netdev}
OVS includes a userspace datapath interface (dpif) implemenation, called dpif-netdev.
Unlike the dpif-netlink which talks to the Linux Open vSwitch kernel
module through netlink API, the dpif-netdev works completely in userspace.
As a result, dpif-netdev expects to receive and send packets from its userspace
interface.  One major use case of dpif-netdev is OVS-DPDK~\cite{ovsdpdk},
where the packet reception and transmission are all done in DPDK's userspace
library. The dpif-netdev is designed to be agnostic to how the network
device accessing the packets, by an abstraction layer called netdev.
So Packets might come from DPDK packet I/O library,
a Linux AF\_PACKET socket API, or in our case, the AF\_XDP socket interface,
as long as each mechanism implements its own netdev interface.
Once dpif-netdev receives a packet, it follows the same mechanism doing
parse, lookup the flow table, and apply actions to the packet.

Figure~\ref{ovsafxdp} shows the architecture of Userspace OVS with AF\_XDP.
We implement a new netdev type called netdev-afxdp, which receives and transmits
packets using the XSK.
In order for XSK to work, we insert a XDP program and a eBPF map with   
interact with XDP program to forward packets to the AF\_XDP socket.
Once the AF\_XDP netdev receives a packet, it passes the packet to the
dpif-netdev for packet processing.

\subsection{AF\_XDP netdev configuration}
ovs-vswitchd controls and manages the XDP program and maps.
When users attach a AF\_XDP netdev to the OVS bridge, the netdev-afxdp
does the following steps:
We implemnet most of the logic in lib/netdev-linux.c
{
\begin{verbatim}
    ovs-vsctl add-br br0
    ovs-vsctl add-port br0 eth0 -- \
        set int eth0 type="afxdp"
\end{verbatim}
}

\begin{enumerate}
\item Attach a XDP program to the netdev queue:
The XDP program OVS attaches is fixed and consists
of only a few line of code, which receives the packet and
calls the \texttt{bpf\_redirect\_map()} helper function with
the XSK eBPF maps and key equals zero.

\item Create a AF\_XDP socket:
This requires creating the XSK socket by calling socket() syscall,
setting up its RX and TX ring buffer, allocating a UMEM region,
and setting up FILL/COMPLETION ring of the UMEM.

\item Load and configure the XSK eBPF map:
The XSK eBPF maps consists of key value pairs, where key is an u32 index and value is
the netdev's queue ID.  In our case, we simply program one entry to the map with
key equals zero and the file descriptor, fd, of the XSK as its value.
As a result, the XDP program calling \texttt{bpf\_redirect\_map} with key=0 will
forwards the packet to the XSK.

\item Initialize the FILL queue
Before any packet arrival, the rest three queues: RX, TX and COMPLETION
remains the same
\end{enumerate}

When a netdev-afxdp is detached or closed by user,
ovs-vswitchd closes the XSK socket, free its memory, and unload the eBPF program and map.

memory allocation:

% 1 UMEM per XSK per queue, no sharing
For each netdev, ovs-vswitchd attach XSK to each 
For each XSK
memorypool
We implement a UMEM freelist to keep track of the used/unused element

To handle multiple 
\subsection{Packet Reception and Transmission}
\begin{figure}
{\scriptsize
\begin{verbatim}
UMEM: total 8 elements, each has 2K chunk size
                 +-------------------------------+
                 |   |   |   |   |   |   |   |   |
                 +-------------------------------+
            addr:  1   2   3   4   5   6   7  8 
(1) Initial Stat
           -------------------          UMEM freelist
FILL Qu  ...| 1 | 2 | 3 | 4 | ...       {5, 6, 7, 8}
           -------------------
           -------------------
RX ring  ...|   |   |   |   | ...
           -------------------
(2) 4 Packets Arrive
           -------------------          UMEM freelist
FILL Qu  ...|   |   |   |   | ...       {5, 6, 7, 8}
          --------------------
           -------------------
RX ring  ...| 1 | 2 | 3 | 4 | ...
           -------------------
(3) Get 4 free UMEM elemet and refill
           -------------------          UMEM freelist
FILL Qu  ...| 5 | 6 | 7 | 8 | ...       {} 
           -------------------
           -------------------
RX ring  ...| 1 | 2 | 3 | 4 | ...
           -------------------
(4) Pass packets {1, 2, 3, 4} to dpif-netdev
    for parse, lookup, and actions.

(5) Recycle to UMEM freelist, goto step (2)
            +-------------------        UMEM freelist
FILL Qu  ...| 5 | 6 | 7 | 8 | ...       {1, 2, 3, 4}
            +------------------
           -------------------
RX ring  ...|   |   |   |   | ...
           -------------------
\end{verbatim}
}
\vspace{-1.0em}
\caption{An example of how OVS programs the FILL queue and RX ring when
processing incoming AF\_XDP packets.}
\label{afxdprx}
\vspace{-1.0em}
\end{figure}
Figure~\ref{afxdprx} shows how ovs-vswitchd sets up the FILL queue and
the RX ring for receving packets from XSK.  For simplicity to demonstrate
how AF\_XDP works, assume that there are only eight UMEM buffers, with each buffer's
size equals 2KB.  Initially at step 1, ovs-vswitchd fills four empty UMEM elem into
the FILL queue and waits for incoming packets.  When there are incoming packets,
at step 2, the FILL queue's four buffer elems have been consumed and moved
to the RX ring.  In order to keep receiving packets, ovs-vswitchd gets another
four empty UMEM elems from thte UMEM freelist, and fills into FILL queue (step 3).
Then ovs-vswitchd creates the metadata needed for the four packet
buffer, i.e., struct dp\_packet and struct dp\_packet\_batch, and passes to 
the dpif-netdev layer for parse, lookup and action executions.
Finally, when ovs-vswitchd finishes processing the UMEM buffer, a recycle
mechanism is triggered to place these buffer back to UMEM freelist (step 5).
Step 5 makes sure that there are always available elems in FILL queue, so
that the underlying XDP program in kernel can keep placing packets into
the FILL queue while the userspace ovs-vswitchd is processing the previous
received packets.
When step 5 finishes, ovs-vswitchd goes back to step 2, waiting and processing
new packets.

\begin{figure}
{\scriptsize
\begin{verbatim}
 dpif-netdev has 4 packets at UMEM:{1, 2, 3, 4}
 from eth1 and plan to send to eth2.
(1) Get 4 free element from eth2 UMEM, copy packet data to it
(2) Create TX descriptors and place into TX ring
           -------------------        eth2 UMEM freelist
TX ring  ...| 4 | 5 | 6 | 7 | ...       {1, 2, 3, 8}
           -------------------
           -------------------
COMPL q  ...|   |   |   |   | ...
           -------------------
(3) Issue send syscall to eth2's XSK
(4) Packet transmission is completed
           -------------------        eth2 UMEM freelist
TX ring  ...|   |   |   |   | ...       {1, 2, 3, 8}
           -------------------
           -------------------
COMPL q  ...| 4 | 5 | 6 | 7 | ...
           -------------------
(5) Recycle to eth2 UMEM
           -------------------        eth2 UMEM freelist
TX ring  ...|   |   |   |   | ...       {1, 2, 3, 4,
           -------------------           5, 6, 7, 8}
           -------------------
COMPL q  ...|   |   |   |   | ...
           -------------------
(6) Recycle {1, 2, 3, 4} to eth1's UMEM 
\end{verbatim}
}
\vspace{-1.0em}
\caption{An example of how OVS programs the COMPLETION queue and TX ring when
processing incoming AF\_XDP packets from one netdev and sending to another netdev.}
\label{afxdptx}
\vspace{-1.0em}
\end{figure}

Figure~\ref{afxdptx} shows the process for sending packets to the XSK.
Assuming that ovs-vswitch has a a bridge and two ports: eth1 and eth2.
And there is a flow entry showing
\texttt{in\_port=eth1, ..., actions=..., output:eth2}, meaning to forwarding
packets received from eth1 to eth2.
Assuming that both eth1 and eth2 are configured with AF\_XDP support.

Initially, assume ovs-vswitchd receives four packets {1, 2, 3, 4} from eth1's XSK
To send to eth2 using its XSK, ovs-vswitchd first gets four packet buffers,
{4, 5, 6, 7}, from eth2's UMEM freelist and copy the packet data to their
packet buffer in UMEM.  Then TX descriptors for the four packets are
created and placed into eth2's TX ring (step 2).
At step 3, send syscall is issued to signal the kernel to start transmit.
As the send in XSK is asynchronos and send syscall only return return zero when
no error, ovs-vswitch has to poll the COMPLETION queue to make sure these four
packets have been sent out (step 4).
Once the four packets' descriptors shown up at the COMPLETION ring, at step 5,
ovs-vswitchd recycles their UMEM elememnts back to the eth2's UMEM freelist.
In addition, the original four packet buffer from eth1, {1, 2, 3, 4}, is
recycled back to eth1's UMEM freelist.

\subsection{Performance Evaluation}
When sending and receivng packets using AF\_XDP with OVS, two CPUs show
100\% utilitization:
\begin{itemize}
\item ksoftirqd:
\item ovs-vswitchd:
\end{itemize}
PMD
HugePage
UMEM element list
Batching
2-CPU operation
non-blocking send
non-block receive

\mycomment{
/* PMD: Poll modes drivers.  PMD accesses devices via polling to eliminate
 * the performance overhead of interrupt processing.  Therefore netdev can
 * not implement rx-wait for these devices.  dpif-netdev needs to poll
 * these device to check for recv buffer.  pmd-thread does polling for
 * devices assigned to itself.
 *
 * DPDK used PMD for accessing NIC.
 *}


\bibliography{paper.bib}
\bibliographystyle{plain}

\end{document}
